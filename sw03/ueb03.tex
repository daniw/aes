\documentclass[a4,paper,fleqn]{article}

\usepackage{layout}

\title{AES - Übung 3}
\date{10. März 2016}
\author{Markus Birrer \\
        Yannick Inderbitzin\\
        Daniel Winz}

\begin{document}
\maketitle
%\clearpage
\vfill
\tableofcontents
\vfill
\clearpage

\section{Kurzfragen}
\begin{itemize}
\item Wie ist die Energie- und Leistungsdichte eines SCAPs definiert? \\
    Antwort \ldots
\item Wie gross ist eine SCAP-Zeitkonstante typisch? Was last sich davon 
ableiten? \\
    Antwort \ldots
\item Was lässt sich über den Wirkungsgrad von SCAPs generell aussagen? Wie 
gross ist er bezogen auf die Leistung, die durch die 
Leistungsdichte-Definition gegeben ist. Kommentar? \\
    Antwort \ldots
\item Wann macht die Kombination von Supercaps und Batterien Sinn? Was gilt es 
bei der Kombination von Supercaps und Batterien zu beachten? \\
    Antwort \ldots
\item Gibt es Unterschiede bei der Integration von Supercaps gegenüber 
Batterien? \\
    Antwort \ldots
\end{itemize}

\section{Berechnung LiIonen Batterien als Vergleich}
\begin{itemize}
\item Berechnen Sie die Energiedichte. 
\item Berechnen Sie die Leistungsdichte der Ladung und der Entladung. 
Kommentare? 
\item  Berechnen Sie den Wirkungsgrad für nur eine 10C Entladung und separat 
nur für eine 3C Ladung.  Berechnen Sie den Gesamtwirkungsgrad für den 
gesamten Zyklus (Entladung und Ladung).  Annahme: Nur CC Ladeverfahren für 
den ganzen Energieinhalt. 
\end{itemize}

\section{Messung Supercaps}
\subsection{Schlussdiskussion}

\section{Bewertungsbeispiel Akkubohrer}

\end{document}
