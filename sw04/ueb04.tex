\documentclass[a4,paper,fleqn]{article}

\usepackage{layout}

\newcommand{\uuline}[1]{{\underline{\underline{#1}}}}

\title{AES - Übung 4}
\date{10. März 2016}
\author{Markus Birrer \\
        Yannick Inderbitzin\\
        Ervin Mazlagi\'c\\
        Daniel Winz}

\begin{document}
\maketitle
%\clearpage
\vfill
\tableofcontents
\vfill
\clearpage

\section{Abstract}

\section{Prüfstandmessungen Motor}

\subsection{Messergebnisse}
\begin{zebratabular}{p{0.5\textwidth}lll}
    \rowcolor{gray}
    Messung 
        & Leerlauf 
        & Unter Last 
        & Einheit \\
    Geschwindigkeit 
        & 420 
        & 420 
        & RPM \\
    Drehmoment 
        & 0.98 
        & 2.38 
        & Nm \\
    Phasenspannung L1 (peak-peak) 
        & 56.7 
        & 56.8 
        & V \\
    Phasenstrom L1 (peak-peak) 
        & 
        & 3.61 
        & A \\
    Versatz von Phasenspannungsnulldurchgang zu positiver Hallsensorflanke 
        & 1.66 
        & 1.38 
        & ms \\
    Spannung über der externen Spule 
        & 
        & 39.6 
        & V \\
    Frequenz der Spannung 
        & 55.25 
        & 54.9 
        & Hz \\
    Versatz von Phasenspannungsnulldurchgang zu Phasenstromnulldurchgang (Spannung voreilend) 
        & 
        & 1.88 
        & ms \\
\end{zebratabular}

\subsection{Berechnungen}
Polpaarzahl
\[ n = \frac{f_{el}}{f_{mech}} = \frac{55.25}{420 \cdot \frac{1}{60}} = 7.89 \to \uuline{8} \]
Phasenverschiebung zu Hallsensor zu Nulldurchgang
\[ \phi_{offset} = \frac{T_{offset}}{T} = \frac{1.66 ms}{18.0995 ms} = 33^\circ \]

\section{Messfahrt}

\section{Simulation}

\section{Schlussdiskussion}

\end{document}
