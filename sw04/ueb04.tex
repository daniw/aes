\documentclass[a4,paper,fleqn]{article}

\usepackage{layout}

\newcommand{\uuline}[1]{{\underline{\underline{#1}}}}

\title{AES - Übung 4}
\date{10. März 2016}
\author{Markus Birrer \\
        Yannick Inderbitzin\\
        Ervin Mazlagi\'c\\
        Daniel Winz}

\begin{document}
\maketitle
%\clearpage
\vfill
\tableofcontents
\vfill
\clearpage

\section{Abstract}

\section{Prüfstandmessungen Motor}

\subsection{Messergebnisse}
\begin{zebratabular}{p{0.5\textwidth}lll}
    \rowcolor{gray}
    Messung 
        & Leerlauf 
        & Unter Last 
        & Einheit \\
    Geschwindigkeit 
        & 420 
        & 420 
        & RPM \\
    Drehmoment 
        & 0.98 
        & 2.38 
        & Nm \\
    Phasenspannung L1 (peak-peak) 
        & 56.7 
        & 56.8 
        & V \\
    Phasenstrom L1 (peak-peak) 
        & 
        & 3.61 
        & A \\
    Versatz von Phasenspannungsnulldurchgang zu positiver Hallsensorflanke 
        & 1.66 
        & 1.38 
        & ms \\
    Spannung über der externen Spule 
        & 
        & 39.6 
        & V \\
    Frequenz der Spannung 
        & 55.25 
        & 54.9 
        & Hz \\
    Versatz von Phasenspannungsnulldurchgang zu Phasenstromnulldurchgang (Spannung voreilend) 
        & 
        & 1.88 
        & ms \\
\end{zebratabular}

\subsection{Berechnungen}
Polpaarzahl
\[ n = \frac{f_{el}}{f_{mech}} = \frac{55.25}{420 \cdot \frac{1}{60}} = 7.89 \to \uuline{8} \]
Phasenverschiebung zu Hallsensor zu Nulldurchgang
\[ \phi_{offset} = \frac{T_{offset}}{T} = \frac{1.66 ms}{18.0995 ms} = 33^\circ \]

\section{Messfahrt}

\section{Simulation}

\begin{zebratabular}{p{0.5\textwidth}l}
\rowcolor{gray}
Konstanten: &\\
Wirkungsgrad Leistungselektronik: & 0.85 \\
Wirkungsgrad Antriebsmotor: & 0.987 \\
Rollwiderstand ur: & 0.004 \\
Erdanziehungskraft g: & 9.81[m/s^2] \\

Luftwiderstand cw: & 0.6 \\
wirkende Körperfläche: & 0.55 [m^2] \\
Luftdichte: & 1.202 [kg/m^3] \\

Gewicht m: & 100 [kg] \\
Strecke s: & 1480[m] \\


Durchschnittsgeschwindigkeit (anhand GPS-Daten): & 9.06 [m/s] \\

\end{zebratabular}

\subsection{F_Luftwiderstand} \\
F_Luftwiderstand = 0.5*Luftdichte*Luftwiderstand cw*Stirnfläche*Fahrgeschwindigkeit^2 \\
= 16.28 [N] \\
\\
\subsection{Rollwiderstand} \\
F_Rollwiderstand = ur * m * g \\
= 3.92 [N]\\

\subsection{Reibung total}\\
F_Reibungtotal = F_Luftwiderstand + F_Rollwiderstand\\
= 20.20 [N]\\

\subsection{Reibungsenergie}\\
Reibungsenergie = F_Reibungtotal * Strecke\\
= 29901 [Ws]\\

\subsection{Potentielle Energie ohne Rekuperation}\\
Anstieg 1 & 9.3 & [m]\\
Anstieg 2 & 4.3 & [m]\\
Höhendifferenztotal & 13.6 & [m]\\
E_pot = m*a*dh\\
= 13342 [Ws]\\

\subsection{Energie total}\\
E_tot = Reibungsenergie + E_pot\\
= 43243 [Ws]\\

\subsection{benötigte Energie aufgrnd der Wirkungsgrade}\\
E_needed = E_tot/(Wirkungsgrad Leistungselektronik*Wirkungsgrad Antriebsmotor)\\
= 51544 [Ws]




\section{Schlussdiskussion}

\end{document}
