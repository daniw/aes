\documentclass[a4,paper,fleqn]{article}

\usepackage{layout}

\newcommand{\uuline}[1]{{\underline{\underline{#1}}}}

\title{AES - Übung 4}
\date{10. März 2016}
\author{Markus Birrer \\
        Yannick Inderbitzin\\
        Ervin Mazlagi\'c\\
        Daniel Winz}

\begin{document}
\maketitle
%\clearpage
\vfill
\tableofcontents
\vfill
\clearpage

\section{Abstract}

\section{Prüfstandmessungen Motor}

\subsection{Messergebnisse}
\begin{zebratabular}{p{0.5\textwidth}lll}
    \rowcolor{gray}
    Messung 
        & Leerlauf 
        & Unter Last 
        & Einheit \\
    Geschwindigkeit 
        & 420 
        & 420 
        & RPM \\
    Drehmoment 
        & 0.98 
        & 2.38 
        & Nm \\
    Phasenspannung L1 (peak-peak) 
        & 56.7 
        & 56.8 
        & V \\
    Phasenstrom L1 (peak-peak) 
        & 
        & 3.61 
        & A \\
    Versatz von Phasenspannungsnulldurchgang zu positiver Hallsensorflanke 
        & 1.66 
        & 1.38 
        & ms \\
    Spannung über der externen Spule 
        & 
        & 39.6 
        & V \\
    Frequenz der Spannung 
        & 55.25 
        & 54.9 
        & Hz \\
    Versatz von Phasenspannungsnulldurchgang zu Phasenstromnulldurchgang (Spannung voreilend) 
        & 
        & 1.88 
        & ms \\
\end{zebratabular}

\subsection{Berechnungen}
Polpaarzahl
\[ n = \frac{f_{el}}{f_{mech}} = \frac{55.25}{420 \cdot \frac{1}{60}} = 7.89 \to \uuline{8} \]
Phasenverschiebung Hallsensor zu Nulldurchgang
\[ \varphi_{offset} = \frac{T_{offset}}{T} = \frac{1.66 ms}{18.0995 ms} = 33^\circ \]

\section{Gesamtantriebssyteme und Simulationen: Messung an einem Ebike Antrieb}

\subsection{Energiemessfahrt mit dem Stromer-Ebike}
Die Strecke rund um das Areal der T\&A wurde mit dem E-Bike abgefahren. Dabei 
wurde kontinuierlich die Spannung und den Strom zwischen dem Akku und dem 
Kontroller gemessen und geloggt. Aus diesen Daten kann in etwa berechnen 
werden, wie viel Energie für die Probefahrt vom Fahrer aufgewendet wurde und 
wie viel vom Motor geleistet wurde.

\noindent
Um den Strom zu messen, wurde eine Stromzange in die Leitung zwischen Akku und 
Kontroller gehängt. Ebenfalls wird bei den Eingangsbuchsen des Akkus an der 
Box die Spannung abgegriffen. Die Messleitungen führen zu je einem 
MetraHit-Loggerät, welche wiederum auf je ein MetraHit-Multimeter aufgesteckt 
sind. Die beiden Loggeräte werden zusammengesteckt und über einen USB-RS232 
Wandler mit einem Laptop verbunden. Auf dem Laptop ist das Tool MetraWin 10 
installiert, über welches das Datenlogging läuft. Der Laptop wurde zusammen 
mit den Messgeräten in einen Rucksack gepackt, welcher während der Fahrt vom 
Fahrer mitgeführt wurde.

\subsection{Berechnung der Energie aus Antrieb}
Aus den geloggten Daten (siehe Anhang) kann die benötigte Energie des Motors 
während der Testfahrt wie folgt berechnet werden:
\[ E_{tot_{Motor}} = \sum\limits_{t_{start}}^{t_{end}} P dt 
= \sum\limits_{t_{start}}^{t_{end}}  U \cdot I \cdot dt \]
Das Ergebnis lautet: 
\[ E_{tot_{Motor}} = 76'687 Ws = 21.3 Wh \]

\subsection{Berechnung der Energie aus GPS-Daten}
\[ E_{tot_{GPS}} = \sum\limits_{t_{start}}^{t_{end}} P dt 
= \sum\limits_{t_{start}}^{t_{end}} \frac{1}{2} \cdot m \cdot v^2 dt
= \sum\limits_{t_{start}}^{t_{end}} \frac{1}{2} \cdot m \cdot \frac{s^2}{dt} \]
Das Ergebnis lautet:
\[ E_{tot_{GPS}} = 80'266Ws = 22.3Wh \]

\subsection{Diskussion}
Vergleicht man die beiden Ergebnisse, fällt auf, dass sie sehr nahe beieinander 
liegen. Die Differenz zwischen der benötigten Energie des Motors und der 
berechneten Energie aus den GPS Daten beträgt genau eine Wattstunde. Dies ist 
die Energie, die der Fahrer selbst aufgewendet hat, um das Ebike zu 
beschleunigen. Der Fahrer hat nach eigenen Angaben nur sehr selten in die 
Pedalen getreten, da das Ebike über einen Handgasgriff verfügte. Das Treten in 
die Pedale war also nicht unbedingt nötig. Berücksichtigt man diese Angaben 
ist das Ergebnis der Testfahrt plausibel.

\section{Simulation}

\begin{zebratabular}{p{0.5\textwidth}l}
\rowcolor{gray}
Konstanten: &\\
Wirkungsgrad Leistungselektronik: & 0.85 \\
Wirkungsgrad Antriebsmotor: & 0.987 \\
Rollwiderstand ur: & 0.004 \\
Erdanziehungskraft g: & 9.81[m/s$^2$] \\
Luftwiderstand cw: & 0.6 \\
wirkende Körperfläche: & 0.55 [m$^2$] \\
Luftdichte: & 1.202 [kg/m$^3$] \\
Gewicht m: & 100 [kg] \\
Strecke s: & 1480[m] \\
Durchschnittsgeschwindigkeit (anhand GPS-Daten): & 9.06 [m/s] \\
\end{zebratabular}

\subsection{Luftwiderstand} 
\[ F_{Luftwiderstand} = 0.5 \cdot Luftdichte \cdot Luftwiderstand cw \cdot Stirnfläche \cdot Fahrgeschwindigkeit^2 = 16.28 [N] \]

\subsection{Rollwiderstand} 
\[ F_{Rollwiderstand} = ur  \cdot  m  \cdot  g = 3.92 [N]\]

\subsection{Reibung total}
\[ F_{Reibungtotal} = F_{Luftwiderstand} + F_{Rollwiderstand} = 20.20 [N]\]

\subsection{Reibungsenergie}
\[ Reibungsenergie = F_{Reibungtotal}  \cdot  Strecke = 29901 [Ws]\]

\subsection{Potentielle Energie ohne Rekuperation}
\begin{tabular}{ll}
Anstieg 1: & 9.3 [m]\\
Anstieg 2: & 4.3 [m]\\
Höhendifferenztotal & 13.6 [m]\\
\end{tabular}
\[ E_{pot} = m \cdot a \cdot dh = 13342 [Ws]\]

\subsection{Energie total}
\[ E_{tot} = Reibungsenergie + E_{pot} = 43243 [Ws]\]

\subsection{Benötigte Energie aufgrund der Wirkungsgrade}
\[ E_{needed} = \frac{E_{tot}}{(Wirkungsgrad Leistungselektronik \cdot Wirkungsgrad Antriebsmotor)} = 51544 [Ws] \]

\section{Schlussdiskussion}

\clearpage
\begin{appendix}
\section{Daten Solarpanels}
\begin{zebratabular}{lllll}
\rowcolor{gray}
Solarmodul &
    PV\_A &
    PV\_B &
    PV\_C &
    PV\_D \\
Aufbau &
    Polykristallin &
    Monokristallin &
    Monokristallin &
    Monokristallin \\
Herkunft &
    CH &
    CH &
    China &
    Japan \\
$P_G$ $[W_p]$ &
    240 &
    245 &
    260 &
    240 \\
Masse $[mm]$ &
    992 $\times$ 1650 x 45 &
    991 $\times$ 1656 x 65 &
    992 $\times$ 1636 x 45 &
    798 $\times$ 1580 x 35 \\
$U_{mpp}$ $[V]$ &
    30 &
    30.2 &
    30.6 &
    43.7 \\
$U_{OC}$ $[V]$ &
    37.2 &
    36.9 &
    37.3 &
    52.4 \\
$\Delta U_{OC}$ $[\%/^\circ C]$ &
    -0.33 &
    -0.36 &
    -0.31 &
    -0.25 \\
$I_{mpp}$ $[A]$ &
    8 &
    8.1 &
    8.5 &
    5.51 \\
$I_{SC}$ $[A]$ &
    8.65 &
    8.6 &
    9.18 &
    5.85 \\
$\eta$ $[\%]$ &
    14.7 &
    14.9 &
    16 &
    19 \\
Preis $[CHF]$ &
    242 &
    333 &
    224 &
    589 \\
\end{zebratabular}

\section{Daten Wechselrichter}
\begin{zebratabular}{lllll}
\rowcolor{gray}
Solarmodul &
    WR\_A &
    WR\_B &
    WR\_C &
    WR\_D \\
Beschreibung &
    Zentral-WR gross &
    Zentral-WR klein &
    Multistring WR 2 Strings &
    Multistring WR 3 Strings \\
$P_W$ $[kW]$ &
    80 &
    35 &
    10 &
    15 \\
$P_{G_{max}}$ pro String $[kW]$ &
    k. A. &
    k. A. &
    9 &
    9 \\
$U_{min}$ für $P_W$ $[V]$ &
    k. A. (430?) &
    k. A. (400?) &
    290 &
    320 \\
$U_{MPP}$ Bereich $[V]$ &
    430 \ldots 800 &
    400 \ldots 800 &
    250 \ldots 750 &
    250 \ldots 750 \\
$U_{OC_{max}}$ $[V]$ &
    900 &
    900 &
    900 &
    900 \\
$I_{W_{max}}$ $[A]$ &
    180 &
    78 &
    18 &
    16 \\
$\eta_{WR}$ $[\%]$ &
    95.5 &
    96.1 &
    97.5 &
    97.5 \\
Preis $[CHF]$ &
    28'910 &
    11'392 &
    3'366 &
    4'248 \\
\end{zebratabular}

\section{Daten Anschlusskasten}
\begin{zebratabular}{llll}
\rowcolor{gray}
Anschlusskasten &
    AK\_1 &
    AK\_2 &
    AK\_3 \\
Art &
    DC, 16 $\to$ 1 &
    DC, 12 $\to$ 1 &
    AC, 3 phasig \\
Preis $[CHF]$ &
    2'530 &
    2'156 &
    1'000 \\
\end{zebratabular}

\clearpage
\section{Matlab Funktion panelize.m}
\label{app_panelize}
\lstsettingmatlab
\lstinputlisting{panelize.m}

\clearpage
\section{Matlab Skript panels.m}
\label{app_panels}
\lstsettingmatlab
\lstinputlisting{panels.m}

\end{appendix}
\end{document}
